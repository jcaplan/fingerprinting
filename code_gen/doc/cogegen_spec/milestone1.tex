%%%%%%%%%%%%%%%%%%%%%%%%%%%%%%%%%%%%%%%%%
% Programming/Coding Assignment
% LaTeX Template
%
% This template has been downloaded from:
% http://www.latextemplates.com
%
% Original author:
% Ted Pavlic (http://www.tedpavlic.com)
%
% Note:
% The \lipsum[#] commands throughout this template generate dummy text
% to fill the template out. These commands should all be removed when 
% writing assignment content.
%
% This template uses a Perl script as an example snippet of code, most other
% languages are also usable. Configure them in the "CODE INCLUSION 
% CONFIGURATION" section.
%
%%%%%%%%%%%%%%%%%%%%%%%%%%%%%%%%%%%%%%%%%

%----------------------------------------------------------------------------------------
%	PACKAGES AND OTHER DOCUMENT CONFIGURATIONS
%----------------------------------------------------------------------------------------

\documentclass{article}

\usepackage{fancyhdr} % Required for custom headers
\usepackage{lastpage} % Required to determine the last page for the footer
\usepackage{extramarks} % Required for headers and footers
\usepackage[usenames,dvipsnames]{color} % Required for custom colors
\usepackage{graphicx} % Required to insert images
\usepackage{listings} % Required for insertion of code
\usepackage{courier} % Required for the courier font
\usepackage{lipsum} % Used for inserting dummy 'Lorem ipsum' text into the template

\usepackage{paralist}
% Margins
\topmargin=-0.45in
\evensidemargin=0in
\oddsidemargin=0in
\textwidth=6.5in
\textheight=9.0in
\headsep=0.25in

\linespread{1.1} % Line spacing

% Set up the header and footer
\pagestyle{fancy}
\lhead{\hmwkAuthorNameShort} % Top left header
\chead{\hmwkTitle} % Top center head
\rhead{} % Top right header
\lfoot{} % Bottom left footer
\cfoot{} % Bottom center footer
\rfoot{Page\ \thepage\ of\ \protect\pageref{LastPage}} % Bottom right footer
\renewcommand\headrulewidth{0.4pt} % Size of the header rule
\renewcommand\footrulewidth{0.4pt} % Size of the footer rule

%\setlength\parindent{0pt} % Removes all indentation from paragraphs

%----------------------------------------------------------------------------------------
%	CODE INCLUSION CONFIGURATION
%----------------------------------------------------------------------------------------
\definecolor{mygreen}{rgb}{0,0.6,0}
\definecolor{mygray}{rgb}{0.5,0.5,0.5}
\definecolor{mymauve}{rgb}{0.58,0,0.82}

\lstset{ %
  backgroundcolor=\color{white},   % choose the background color; you must add \usepackage{color} or \usepackage{xcolor}
  basicstyle=\footnotesize,        % the size of the fonts that are used for the code
  breakatwhitespace=false,         % sets if automatic breaks should only happen at whitespace
  breaklines=true,                 % sets automatic line breaking
  captionpos=b,                    % sets the caption-position to bottom
  commentstyle=\color{mygreen},    % comment style
  deletekeywords={...},            % if you want to delete keywords from the given language
  escapeinside={\%*}{*)},          % if you want to add LaTeX within your code
  extendedchars=true,              % lets you use non-ASCII characters; for 8-bits encodings only, does not work with UTF-8
  frame=single,                    % adds a frame around the code
  keepspaces=true,                 % keeps spaces in text, useful for keeping indentation of code (possibly needs columns=flexible)
  keywordstyle=\color{blue},       % keyword style
  language=Octave,                 % the language of the code
  morekeywords={*,...},            % if you want to add more keywords to the set
  numbers=left,                    % where to put the line-numbers; possible values are (none, left, right)
  numbersep=5pt,                   % how far the line-numbers are from the code
  numberstyle=\tiny\color{mygray}, % the style that is used for the line-numbers
  rulecolor=\color{black},         % if not set, the frame-color may be changed on line-breaks within not-black text (e.g. comments (green here))
  showspaces=false,                % show spaces everywhere adding particular underscores; it overrides 'showstringspaces'
  showstringspaces=false,          % underline spaces within strings only
  showtabs=false,                  % show tabs within strings adding particular underscores
  stepnumber=2,                    % the step between two line-numbers. If it's 1, each line will be numbered
  stringstyle=\color{mymauve},     % string literal style
  tabsize=2,                       % sets default tabsize to 2 spaces
  title=\lstname                   % show the filename of files included with \lstinputlisting; also try caption instead of title
}


\newcommand{\includecode}[2]{\lstinputlisting[caption=#2,captionpos=t,language=C]{code/#1}}


%----------------------------------------------------------------------------------------
%	NAME AND CLASS SECTION
%----------------------------------------------------------------------------------------

\newcommand{\hmwkTitle}{Nios Code Generation Specification} % Assignment title
\newcommand{\hmwkDueDate}{\today} % Due date
\newcommand{\hmwkClass}{} % Course/class
\newcommand{\hmwkClassTime}{} % Class/lecture time
\newcommand{\hmwkClassInstructor}{} % Teacher/lecturer
\newcommand{\hmwkAuthorName}{Jonah Caplan} % Your name
\newcommand{\hmwkAuthorNameShort}{Caplan} % Your name

%----------------------------------------------------------------------------------------
%	TITLE PAGE
%----------------------------------------------------------------------------------------

\title{
\vspace{2in}
\textmd{\textbf{\hmwkTitle}}\\
\vspace{3in}
}

\author{\textbf{\hmwkAuthorName}}
\date{} % Insert date here if you want it to appear below your name



%----------------------------------------------------------------------------------------

\begin{document}

\maketitle
\thispagestyle{empty}
\newpage
\setcounter{page}{1}

%------------------------------------------------------------------------------------
\section{Introduction}

The aim of this project is to develop an infrastructure for the automatic generation of C code for multicore Nios systems. We assume that an external model based design approach is used to generate control algorithms and that C code for these computations have already been generated in separate files (e.g. Simulink). The purpose of this tool is to efficiently map the control system onto an arbitrary platform while taking into account non-functional requirements such as deadlines, data flow, and criticality. The user must only specify the requirements for the system at a high level of abstraction and all intermdeiate code will be automatically generated. 

This tool will not initially be geared towards solving codesign problems. We will assume a static hardware platform and ocnsider changing software requirements only. In order to facilitate later expansions, the tool will require the specification of the hardware in terms of generic model parameters. The task-mapping procedure will be platform independent and solve the problem generically even if we do not currently take advantage of this feature.

This document will provide the specification for the currently supported hardware models, the platform built from these components currently under study, the application models for analysis of software requirements, the mapping procedure combining both hardware and software models to generate a schedule, and the abstract template requirements for code generation.

\section{Hardware Model}
Hardware models are specified using an object-oriented semantics. The system is divided into hierarchical levels that are interpreted using static scoping rules to aid in the specification of larger models. 

The cost in time of transmitting over communication channel models is omitted at this stage beyond specifying the connections between elements. Issues related to resource arbitration and interference are also not considered. The underlying infrastructure is assumed to be sufficiently quick and deterministic.

There are four categories of hardware models in the system: processor cores, peripheral cores, and memory and buses.

\subsection{Processor cores}
All processor cores are assumed to operate at the same clock frequency. Their parameters are determined during hardware design and are not altered during software mapping. They can be extracted from the \emph{system.h} file generated from the {.sopcinfo} file by the Nios IDE during BSP generation.

Processor parameters are:
\begin{enumerate}
\item \emph{Fault-tolerant}: While we do not have access to safety-critical Nios licenses, they do exist. We assume that a core can be designated as fault tolerant (FT) and that a cost is associated with fault tolerance (due to licensing, size, resource utilization, power consumption as appropriate for the scenario) and that it is therefore necessary to have fewer FT cores in the system.

\item \emph{Scratchpad}: 
The processor must have a scratchpad in order to use fingerprinting as an error detection mechanism under our current implementation. The relevant parameters for the scratchpad will be listed separately.

\item \emph{Timer}: 
The system timer will dictate the minimum period for events in the system. The timer period is statically assigned when specifying the hardware design in QSYS.

\item \emph{Fingerprint Unit}: 
Fingerprint units will be necessary to allow the monitoring of processing cores (PC), i.e. those cores lacking FT capabilities, by a FT core.
 
\item \emph{Data and instruction cache}: 
It may be necessary to disable data and/or instruction caches while executing critical tasks. It must be known if the processor is equipped with either.

\item \emph{DMA}:
A single channel DMA will be used to shuttle critical data in and out of the scratchpads.

\item \emph{MPU}:
An MPU will be required to ensure that each core is unable to maintain partitions between each core.

\item \emph{Shared memory}:
Shared memory space will be required to load instructions

\item \emph{Interrupt signals}:
The processor model must specify the actively connected interrupt signals.

\end{enumerate}

This is a very general model of the processor. The mapping of tasks to cores will take place considering a very high level model of the system. The lower level parameters will only be used for code generation purposes. For the processor model, it is sufficient to consider whether or not each of these components are available. The details of each component will be hidden from the mapping problem at this higher level of abstraction.

\subsection{Memory}
Local scratchpads will be required for each core as well as shared main memory. A memory is defined simply as a start and end address. Memory latencies are not modelled. A memory model will also keep track of what other modules are connected to it.

There will be two partitioned sections of shared main memory. One to access common functions for redundant task executions and another for message passing between cores.

\subsection{Peripherals}
Certain details about the peripherals such as control registers will be entirely encapsulated in the template objects for the code generation phase. The higher level concerns that will dictate how these registers are sit will be included in the object representation of each peripheral. Mappings from the higher level concerns to code generation rules will be specified.

\subsubsection{Timer}
The frequency of the system level timer must be known in order to define the minimum resolution of time in the application model. Preexisting drivers exist and macros are generated by the Nios IDE to manage the timer. All relevant macros can be extracted from the generated code.

\subsubsection{Fingerprint Unit} 
The fingerprint unit has the following features: maximum stack depth, statically set during hardware design, and block length size. The setting of

\subsubsection{Memory Protection Unit}

\subsubsection{Comparator}

\subsubsection{DMA}

\subsubsection{$\mu$TLB}

\section{Application Model}
\subsection{Synchronous Languages - Building off the past}


\end{document}