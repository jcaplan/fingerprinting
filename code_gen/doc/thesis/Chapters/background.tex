% Chapter Template

\chapter{Background} % Main chapter title

\label{c:background} % Change X to a consecutive number; for referencing this chapter elsewhere, use \ref{ChapterX}

%----------------------------------------------------------------------------------------
%	SECTION 1
%----------------------------------------------------------------------------------------

\section{Mixed Criticality}
\addfigure{0.7}{responsetime.pdf}{Example of priority inversion in mixed criticality system using rate monotonic scheduling.}{f:responsetime}
AMC-rtb \cite{baruah2011response}:

In the LO mode:
\begin{equation}
R_i^{(LO)}= C_i(LO)+\sum_{j \in hp(i)}\Big\lceil\frac{R_i^{(LO)}}{T_j}\Big\rceil \cdot C_j(LO)
\label{eq:mode1}
\end{equation}
In the HI mode:
\begin{equation}
\begin{aligned}
R_i^{(HI)} &  = C_i(HI)+\sum_{j \in hpH(i)}\Big\lceil\frac{R_i^{(HI)}}{T_j}\Big\rceil \cdot C_j(HI) \\
&  +\sum_{k \in hpL(i)}\Big\lceil\frac{R_i^{(LO)}}{T_k}\Big\rceil \cdot C_k(LO)
\end{aligned}
\end{equation}



\section{On-Demand Redundancy}
Figures \ref{f:lockstep} and \ref{f:odr}.
\addfigure{0.7}{lockstep.pdf}{Example of a multicore system with lockstep execution}{f:lockstep}
\addfigure{0.7}{odr.pdf}{Example of multicore system with on-demand redundancy}{f:odr}


\section{Fingerprinting with Nios Cores}

FPGA prototype in Figure~\ref{f:platform-arch}.

\addfigure{1.2}{arch.pdf}{Platform Architecture}{f:platform-arch}
 \addfigure{0.7}{qsort.pdf}{Hamming distance for qsort benchmark}{f:qsort}

\addfigure{0.7}{qsort_d.pdf}{Cumulative distribution for hamming distance for qsort benchmark}{f:qsort-d}
 
\section{Virtual Platform Model}

Corresponding virtual platform. All work in this thesis only validated on virtual platform.

\section{Code Generation}
Simulink, partioning, code extraction, generalized solution.



