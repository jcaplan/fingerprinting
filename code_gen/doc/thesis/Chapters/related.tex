% Chapter Template

\chapter{Related Work} % Main chapter title

\label{c:related} % Change X to a consecutive number; for referencing this chapter elsewhere, use \ref{ChapterX}

%----------------------------------------------------------------------------------------
%	SECTION 1
%----------------------------------------------------------------------------------------

	This thesis presents an analysis and code generation framework to port Simulink generated control algorithms to a multicore system. 
	The framework combines elements of static program analysis, fault tolerant embedded computing, design space exploration, and schedulability analysis,
	This chapter briefly discusses work related to each of these topics that influenced this work.

\section{Real-Time Code Generation}

	Many recent projects explore code generation for real-time multicore systems. 
	Several groups have demonstrated integrated frameworks that allows specification of both a hardware and software model and generates both hardware and software code \cite{goossens2013compsoc,gauthier2001automatic}. 
	Strategies for code generation in real-time systems also exist that use platform specific features to address thread replication and fault tolerance \cite{huang2012framework} for a static hardware platform. 
	This thesis implements a similar framework with coarse grained hardware customization (mainly number of cores and size of memory) that generates code for a pre-existing multicore architecture with a focus on explicit integration with Simulink.
	
\section{Design Space Exploration}

Mapping strategies and automated design space exploration for multicore systems are explored in \cite{bolchini2013reliability}, and \cite{kang2014static}. Recent work has also examined porting of multi-rate synchronous languages to multicore systems \cite{puffitsch2013mapping}. We have proposed a four-mode response time analysis for mixed-criticality fault tolerant systems and developed a similar design space exploration technique based on genetic algorithms using our analysis~\cite{albayati2016modes}.

\section{Schedulability Analysis}

The schedulability analysis in this thesis builds off of the AMC-rtb response time analysis for mixed criticality systems~\cite{baruah2011response} which extends classical rate monotonic scheduling. One main alternative is earliest-deadline-first (EDF) global scheduling for mixed criticality systems~\cite{kim2013novel}. The accuracy of the EDF schedulability analysis compared to a real many-core implementation was tested in \cite{sigrist2015mixed}. It was found that the models are inaccurate and that many of the model assumptions are extremely optimistic (up to 97\% of systems that passed analysis failed on real hardware). Work moving forward will depend on closing the gap between the models and systems and working non-ideal considerations in an integrated framework. This thesis provides a tool for rapid prototyping of software systems to explore discrepancies between the abstract analysis, virtual model and physical implementation.

% An effective profiling tool will be critical to reducing time-consuming iterations and reducing at least one key source of error in the mapping and scheduling analysis.

\section{WCET Estimation}

	Implicit path enumeration technique (IPET) is a standard method of analysis to determine the worst case execution time of a program from its control flow graph \cite{li1995performance}. 
	The IPET framework developed for this thesis is inspired by the open source tool Heptane~\cite{heptane}. 
	The static analysis on loops is based on the standard \emph{Modern Compiler Implementation}~\cite{andrew2002modern}. 
	The implementation of the framework draws inspiration from the open source McSAF static analysis framework for Matlab~\cite{mcsaf}.

\section{On-Demand Redundancy}
	The target architecture and semantics of the generated code in this thesis relies on fingerprinting based on-demand redundancy using thread-level replication.
 	Fingerprinting was introduced in \cite{Smolens:04} as a method of detecting errors. 
 	On-demand redundancy \cite{Meyer:CASES11,fu2013demand} and dynamic core coupling \cite{lafrieda2007utilizing} discuss potential performance gains arising from non-lockstep based architectures for mixed-criticality applications.  
 	The L4.fiasco microkernel also uses fingerprinting to monitor redundant threads \cite{dobel2012operating} and propose a similar architecture with a \emph{trusted computing base} \cite{dobel2012watches}. 

\section{Alternative Methods of Fault-Tolerance}
Alternative architectures exist to detect and correct errors in multicore systems. Moma \cite{ferreira2014adaptive} is a proposed architecture that uses replication in the pipeline to detect errors at the micro-architectural level. The compiler forces live state out of registers at the end of all basic blocks to reduce opportunities for errors to manifest in the registers themselves and a fault tolerant matrix multiplication accelerator accompanies a RISC core to boost performance. Nostradamus is a low cost single core hardened architecture that employs various techniques to protect the pipeline \cite{nathan2014nostradamus}. Software implemented fault tolerance applies compiler transformations to insert replication and self-testing at the assembly level \cite{reis2005swift}.