% Chapter Template

\chapter{Conclusions and Future Work} 

\label{c:concl} 

	There are severe limitations across the framework that may be improved in future work.
	The static analysis tool lacks any inter-procedural analysis which prevents it from tackling many practical benchmarks.
	The IPET analysis requires some form of microarchitectural modelling to generate cycle accurate estimates.
	The schedulability analysis requires that all interrupt handlers be modelled as tasks as well as the monitor DMA task and data transfers.
	The entire system does not allow for precedence-relations between tasks or data exchange. 
	Furthermore, the reading of sensor data and writing of final answers to actuators is not modelled (any practical solution would be board specific and involve a fair bit of extra hardware design - beyond the scope of this thesis).
	Code generation is limited to the three core system and does not support TMR or PR.
	Other fallback features such as watchdog timers and restarting in case of fatal errors such as null pointers have not been implemented.
	Algorithms which employ table lookups cannot use ODR because the tables generally will not fit in the scratchpad.
	A task may not acquire more than one page of scratchpad memory even if it is available.
	The schedulability analysis does not consider the scratchpad as a limited resource.
	The genetic algorithm based DSE is not a practical solution for heavily loaded systems as the search times are excessive and inconsistent given the goal of rapid prototyping. 
	It is also possible that the monitor code suffers from race conditions that do not manifest in the virtual platform. 
	Considerable time will be required to set up a robust debugging environment prior to validation on the FPGA.
	


	With these limitations acknowledged, this thesis has presented a framework to automate the porting of Simulink generated control loops to a multicore platform with on-demand redundancy.
	The main goal of the project was to develop starting points for the three main components in the system in order to facilitate future work of a more inter-disciplinary (co-design) nature. 
	Static analysis tools provide a baseline for WCET estimation and memory requirements. 
	A mapping and scheduling tool provides analysis to determine an optimal allocation on the system.
	Finally, the remaining source code is generated for the platform as well as scripts to facilitate user customizations.
	Hopefully, non-trivial control systems may be demonstrated on the platform in the future.
		