% Appendix A

\chapter{User Configuration File} % Main appendix title

\label{A:ConfigFile} % For referencing this appendix elsewhere, use \ref{AppendixA}


The configuration file provided to the program must be able to express the basic flow as well as leave open the possibility of integrating extensions in the future. The syntax should be both easy to read for the user while also being easy to parse.

First, the functions must be specified. Each function must have a root name, a directory location as well as a constructor for global variables. The naming conventions for function names will follow the convention for Simulink generated code. The folder must contain the \texttt{ert\_main.c} file that provides variable declarations and initialization code.

The configuration file will be specified through examples. First, the function declarations:

\begin{lstlisting}[caption={Function configuration syntax},label=l:config-function,language=bash]
#periodic tasks, some are critical, all independent (no dataflow between tasks)


#function list must come first!!

#Comments start with #
# -c for critical task
# -T for task period in ms
# -E for event driven, runs when all the inputs are available
# -S for stacksize, if S is not present then assume profiler was used
# 	if profiler was used: should have a profile.log file in the function folders
<FUNCTIONLIST>
-rootdir /home/jonah/fingerprinting/automotive_control/CompiledC/for_loops
-name for_loop_100000_0    -subdir for_loop_100000_0           -T  30   -c  -Priority 2	
-name for_loop_50000_50000 -subdir for_loop_50000_50000		   -T  30       -Priority 1 -printRuntime -addPreamble preamble.txt
</FUNCTIONLIST>

#preamble -> in subdir folder, contains code to go before while loop in task, used for threadsafe newlib config, should not have extension .c or .h
#printRuntime adds diagnostic printf run time for the given task

<DATAFLOW>
</DATAFLOW>


<PLATFORM>
-usedefault
</PLATFORM>

<MAPPING>
for_loop_100000_0 cpu0 cpu1
for_loop_50000_50000 cpu0
</MAPPING>

#optional - skip stack profiling if included
<STACK_PROFILE>
for_loop_100000_0 80
for_loop_50000_50000 80
</STACK_PROFILE>


#optional - skip wcet profiling if included
<WCET_PROFILE>
for_loop_50000_50000 1600035
for_loop_100000_0 1600004 2400006
</WCET_PROFILE>
\end{lstlisting}


Whitespace is ignored except for line breaks. Each top level function must have a unique name and only be instantiated once.
