% Appendix A

\chapter{User Configuration File} % Main appendix title

\label{A:ConfigFile} % For referencing this appendix elsewhere, use \ref{AppendixA}


The configuration file provided to the program must be able to express the basic flow as well as leave open the possibility of integrating extensions in the future. The syntax should be both easy to read for the user while also being easy to parse.

First, the functions must be specified. Each function must have a root name, a directory location as well as a constructor for global variables. The naming conventions for function names will follow the convention for Simulink generated code. The folder must contain the \texttt{ert\_main.c} file that provides variable declarations and initialization code.

The configuration file will be specified through examples. First, the function declarations:

\includecode{configfile.sh}{l:config-function}{Configuration file example}{bash}

Whitespace is ignored except for line breaks. Each top level function must have a unique name and only be instantiated once.
